\section{Ruoli di Progetto}

Per tutta la durata del progetto, i componenti del gruppo andranno ad assumere diversi ruoli, rappresentativi di figure aziendali specializzate, indispensabili per una buona organizzazione interna e per il buon esito del lavoro. Ogni membro dovrà, a rotazione, assumere \textit{almeno una volta} ogni ruolo, come specificato nei vincoli di organigramma. Per evitare possibili conflitti causati dalla rotazione dei ruoli, le attività
principali assegnabili a specifici ruoli sono pianificate nel \PianoDiProgetto{}. Ogni componente del
gruppo è tenuto a svolgere le attività assegnategli e a rispettare il ruolo che ne consegue. Il \textit{Responsabile
di Progetto} ha il compito di fare rispettare i ruoli assegnati durante le attività, mentre il \textit{Verificatore}
deve individuare le possibili incongruenze tra i ruoli e le modifiche registrate nei diari delle modifiche.
Segue quindi una breve descrizione dei vari ruoli e delle varie mansioni che dovranno adempiere.

\subsection{Amministratore}
L'\textit{Amministratore} equipaggia, organizza e gestisce l'ambiente di lavoro e di produzione. Collabora con il \textit{Responsabile di Progetto} alla stesura del \textit{Piano di Progetto} e redige le \textit{Norme di Progetto}. Le responsabilità che andrà ad assumere sono:
\begin{itemize}
\item Attuare le scelte tecnologiche concordate con il \textit{Responsabile di Progetto};
\item Controllare versioni e configurazioni del prodotto;
\item Risolvere i problemi legati alla gestione dei processi.
\end{itemize}

\subsection{Analista}
L'\textit{Analista} è responsabile dell'attività di analisi. Per poter giungere correttamente a tale scopo deve comprendere a fondo il dominio applicativo. Redige lo \textit{Studio di Fattibilità} e l'\textit{Analisi dei Requisiti}, ovvero una specifica di progetto che cerca di comprenderne i vincoli e i rischi tecnologici. Non si occupa di trovare una soluzione al problema posto dal capitolato di progetto, ma cerca di estrarre il maggior numero di requisiti.

\subsection{Progettista}
Il \textit{Progettista} è responsabile dell'attività di progettazione, ha una profonda conoscenza delle tecnologie che si andranno ad utilizzare ed è aggiornato alle ultime novità proposte da queste. Ha il compito di trovare una soluzione dati i requisiti forniti dall'\textit{Analista}, e, una volta compiuta la scelta, se ne assume la piena responsabilità. Redige la \textit{Specifica Tecnica}, la \textit{Definizione di Prodotto} e la parte programmatica del \textit{Piano di Qualifica}.

\subsection{Programmatore}
Il \textit{Programmatore} ha responsabilità circoscritte: si occupa dell'attività di codifica nel rispetto delle \textit{Norme di Progetto} e di quanto stabilito dal \textit{Progettista}, senza spazio per fantasie o idee personali.

\subsection{Responsabile di Progetto}
Il \textit{Responsabile} ha sempre l'ultima voce in capitolo per quanto concerne le decisioni sul progetto e ne è il responsabile ultimo dei risultati. Approva l'emissione di tutti i documenti, redige il \textit{Piano di Progetto} in collaborazione con l'\textit{Amministratore}. Le principali responsabilità che si deve assumere sono le seguenti:
\begin{itemize}
\item Pianificazione e organizzazione dello sviluppo del progetto, stima tempi e costi, e assegnazione delle attività ai componenti del gruppo;
\item Riportare lo stato del progetto ai committenti;
\item Analizzare i rischi che possono incorrere, monitorarli e prendere provvedimenti a riguardo;
\item Stabilire una \glossario{way of working} per ogni componente del gruppo, ai fini di un'influenza positiva delle performance del gruppo.
\end{itemize}

\subsection{Verificatore}
Il \textit{Verificatore} organizza ed attua le attività di verifica e controlla che le attività siano conformi
alle norme. Va a controllare la struttura grammaticale e semantica di ogni documento, andando a correggere eventuali errori. Redige la parte del \textit{Piano di Qualifica} che illustra l'esito e la completezza delle verifiche e delle prove effettuate.